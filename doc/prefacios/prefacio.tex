\thispagestyle{empty}

\begin{center}
{\large\bfseries Memes para todos \\ Plataforma web para la creación, edición y almacenamiento en línea de memes }\\
\end{center}
\begin{center}
Daniel González Serrano\\
\end{center}

%\vspace{0.7cm}

\vspace{0.5cm}
\noindent\textbf{Palabras clave}: \textit{software libre}, \textit{memes}, \textit{edición de memes}, \textit{aplicación web de memes}, \textit{enfoques ágiles}, \textit{next.js 15}, \textit{fabric.js}, \textit{canvas} \textit{prisma} \textit{orm} \textit{postgresql} \textit{git} \textit{github} \textit{node.js} \textit{react} \textit{typescript} \textit{tailwindcss} \textit{vercel}

\vspace{0.7cm}

\noindent\textbf{Resumen}

El proyecto consiste en el desarrollo de una aplicación web enfocada en la creación, edición y almacenamiento de memes, destacándose por su enfoque en la colaboración y la organización de contenidos visuales. Utilizando enfoques ágiles y herramientas de desarrollo como Git y GitHub, la aplicación facilita la colaboración y la edición de memes en un entorno accesible desde cualquier dispositivo. Con tecnologías como Next.js y Fabric.js, la aplicación ofrece una interfaz moderna y funcional que permite crear catálogos de memes, asegurando su fácil acceso y reutilización por parte de los usuarios. Entre los objetivos alcanzados se incluyen la accesibilidad económica, compatibilidad con dispositivos de bajo rendimiento y la implementación de un sistema modular que facilita su actualización y escalabilidad. Este proyecto representa una solución innovadora dentro del campo del almacenamiento y creación colaborativa de contenido visual.
	

\cleardoublepage

\begin{center}
	Daniel González Serrano\\
\end{center}
\vspace{0.5cm}
\noindent\textbf{Keywords}: \textit{free software}, \textit{memes}, \textit{meme editing}, \textit{web meme application}, \textit{agile approaches}, \textit{next.js 15}, \textit{fabric.js}, \textit{canvas} \textit{prisma} \textit{orm} \textit{postgresql} \textit{git} \textit{github} \textit{node.js} \textit{react} \textit{typescript} \textit{tailwindcss} \textit{vercel}
\vspace{0.7cm}

\noindent\textbf{Abstract}

% textidote: ignore begin
The project consists of the development of a web application focused on the creation, editing and storage of memes, with a focus on collaboration and organisation of visual content. Using agile approaches and development tools such as Git and GitHub, the application facilitates the collaboration and editing of memes in an environment accessible from any device. With technologies such as Next.js and Fabric.js, the application offers a modern and functional interface that allows the creation of meme catalogues, ensuring easy access and reuse by users. The objectives achieved include economic accessibility, compatibility with low-performance devices and the implementation of a modular system that facilitates updating and scalability. This project represents an innovative solution in the field of storage and collaborative creation of visual content.
% textidote: ignore end

\cleardoublepage

\thispagestyle{empty}

\noindent\hspace*{-\parindent}\rule[-1ex]{\textwidth}{2pt}\\[4.5ex]

D. \textbf{Juan Julián Merelo Guervós}, profesor del departamento de Arquitectura y Tecnología de Computadores de la Universidad de Granada.

\vspace{0.5cm}

\textbf{Informo:}

\vspace{0.5cm}

Que el presente trabajo, titulado \textit{\textbf{Memes para todos}},
ha sido realizado bajo mi supervisión por \textbf{Daniel González Serrano}, y autorizo la defensa de dicho trabajo ante el tribunal
que corresponda.

\vspace{0.5cm}

Y para que conste, expiden y firman el presente informe en Granada a \today

\vspace{1cm}

\textbf{El director: }

\vspace{5cm}

\noindent \textbf{Juan Julián Merelo Guervós}

\chapter*{Agradecimientos}




