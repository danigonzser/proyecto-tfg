\chapter{Introducción}

Los memes se han convertido en un lenguaje universal en el mundo digital, traspasando fronteras, culturas y generaciones. Desde las conversaciones informales en grupos de chat, hasta en las estrategias de marketing de grandes empresas, los memes son herramientas poderosas para transmitir ideas, emociones y mensajes de forma rápida y efectiva. Sin embargo, a pesar de su uso generalizado, la gestión, centralización y organización de memes sigue siendo un desafío para muchos creadores de contenido, empresas y personas.

\section{Motivación}

Crear y compartir memes es una actividad común, la falta de un sistema centralizado limita su potencial tanto para creadores individuales como para empresas y organizaciones. Esta falta de una solución informática específica se traduce en una pérdida de tiempo y recursos, así como en una limitación en la capacidad de generar contenido relevante y oportuno.

La motivación detrás de este proyecto radica en ofrecer una solución que resuelva problemas y satisfaga las necesidades de diversos usuarios en diferentes contextos. El proyecto "Memes para todos" aspira a proporcionar una plataforma robusta y fácil de usar que simplifique el proceso de creación, almacenamiento y distribución de memes.

\section{Definición del problema}

Dentro de este marco de trabajo, para realizar una mejor de definición del problema, se han identificado una serie de personas con necesidades y objetivos específicos que junto con las historias de usuario describen los problemas que se pretenden resolver con este proyecto.

\subsection{Usuarios identificados}

    \subsubsection{María Rodríguez Espinosa (Creadora de contenido)}

    María Rodríguez Espinosa es una creadora de contenido especializada en memes. Trabaja actualmente para una famosa cadena de comida rápida. Su objetivo principal es mantener comprometida a su audiencia con contenido fresco, joven y relevante.

    \subsubsection{Juan Pérez Ruiz (CTO)}

    Juan Pérez es el CTO de una empresa de marketing digital. Como CTO, es responsable de liderar la estrategia tecnológica de la empresa y garantizar que las soluciones tecnológicas satisfagan las necesidades del negocio y de los clientes. Recientemente, el departamento de marketing le ha expresado la necesidad de un sistema centralizado para agilizar el proceso de generación y gestión de multimedia (principalmente memes).

    \subsubsection{Carlos Sánchez Ruedo (Dueño de un pequeño negocio)}

    Carlos Sánchez es el propietario de un pequeño negocio local especializado en la venta de comida rápida. Inspirado por la exitosa estrategia de KFC de posicionarse como una marca influencer en las redes sociales decide implementar una estrategia similar para promocionar su negocio y aumentar su visibilidad en línea. Sin embargo, debido a las limitaciones de recursos y la falta de experiencia tecnológica no ha dado el paso todavía.

\section{Historias de usuario}

    \subsection{María Rodríguez Espinosa (Creadora de contenido)}

    \begin{enumerate}
        \item [HU01] Como María Rodríguez, creadora de contenido de memes para KFC, necesito poder editar memes y almacenarlos.
        \item [HU02] Como María Rodríguez, creadora de contenido de memes para KFC, siempre he tenido problemas encontrando memes antiguos que he creado.
    \end{enumerate}

    \subsection{Juan Pérez Ruiz (CTO)}

        \begin{enumerate}
            \item [HU02] Como Juan Pérez, CTO, necesito que puedan colaborar varios usuarios en la creación de los memes además de administrar quién del equipo puede ver, editar o eliminar los memes.
        \end{enumerate}

    \subsection{Carlos Sánchez Ruedo (Dueño de un pequeño negocio)}

        \begin{enumerate}
            \item [HU03] Como Carlos Sánchez, dueño de un pequeño negocio, necesito un sistema que se pueda usar desde un pequeño ordenador que tengo en la tienda.
            \item [HU04] Como Carlos Sánchez, dueño de un pequeño negocio, quiero que me facilite la compartición con las redes sociales, puesto que no sé mucho de ellas y no tengo a nadie en mi equipo de marketing.
        \end{enumerate}

\section{Milestones}

Las descripciones de productos concretos o productos mínimos viables (MVP) son una herramienta útil para definir y comunicar el alcance de un proyecto. En este caso, se han identificado los siguientes milestones:

\begin{enumerate}
    \item [M00] - Documentación, planificación y configuración inicial
    \item [M01] - Domain Driven Design
    \item [M02] - Almacenamiento y acceso
    \item [M03] - Búsqueda y gestión
    \item [M04] - Tests y despliegue
    \item [M05] - Seguridad y privacidad
\end{enumerate}