\chapter{Introducción}

¿Alguna vez ha tenido problemas a la hora de encontrar una imagen en su galería?

Si la respuesta es sí y que después de estar guardando contenido en su solución de almacenamiento elegida durante un tiempo, se vuelve insostenible y ya no es nada útil, el problema que se propone resolver este TFG le va a ser de interés. Al igual que usted, hay mucha gente que, debido a la gran cantidad de contenido digital que maneja le es imposible realizar una acción tan fácil (o difícil) como es la de encontrar una imagen.

\section{Contexto y motivación}

En este escenario de generación de contenido digital sin parangón se presenta un desafío para los creadores de contenido: la gestión y almacenamiento eficiente, eficaz
y ágil de dicho contenido. A pesar de contar con diversas plataformas de almacenamiento en la nube y soluciones de gestión de contenidos digitales, la mayoría de
estas soluciones tienen diferentes limitaciones que afectan a la organización, ordenación y accesibilidad de este contenido. La necesidad de una solución que permita
a los usuarios almacenar, organizar y acceder a su amplio repertorio contenido digital de forma eficiente, eficaz y ágil es evidente.

Esta falta de soluciones eficientes, eficaces y ágiles para la gestión y almacenamiento de contenido digital es la motivación principal de este proyecto. Aquí se
presenta una oportunidad significativa para desarrollar un sistema más robusto y adaptado a las necesidades actuales.

Al mejorar la eficiencia, eficacia y agilidad a la hora de la gestión y almacenamiento del contenido digital los usuarios pueden dedicarse a lo que realmente les importa:
la creación de contenido permitiéndoles así dedicarle más tiempo a la creatividad y a la expresión que requiere la creación de contenido digital sin miedo a perder o
no poder acceder a su contenido.

En este contexto y motivación, el presente proyecto se propone abordar estas limitaciones, diseñando y desarrollando un sistema de almacenamiento online que no solo almacene, 
sino que también organice, ordene y facilite el acceso a contenido digital de manera intuitiva y eficiente.

\section{Definición del problema}

Como se ha señalado previamente, la generación masiva de contenido digital ha creado un problema de gestión y almacenamiento del mismo. A pesar de la existencia de
soluciones en la nube y plataformas de gestión de contenido, existen limitaciones a la hora de la gestión y organización de imágenes. La ausencia de una solución que
permita la gestión y organización de imágenes de forma centralizada y eficiente es el problema que se pretende abordar con este proyecto.

\subsection{Usuarios identificados}

    \subsubsection{María Rodríguez Espinosa (Creadora de contenido)}

    María Rodríguez Espinosa es una creadora de contenido especializada en memes. Trabaja actualmente para una famosa cadena de comida rápida. Su objetivo principal es mantener comprometida a su audiencia con contenido fresco, joven y relevante.

    \subsubsection{Juan Pérez Ruiz (CTO)}

    Juan Pérez es el CTO de una empresa de marketing digital. Como CTO, es responsable de liderar la estrategia tecnológica de la empresa y garantizar que las soluciones tecnológicas satisfagan las necesidades del negocio y de los clientes. Recientemente, el departamento de marketing le ha expresado la necesidad de un sistema centralizado para agilizar el proceso de generación y gestión de multimedia (principalmente memes).

    \subsubsection{Carlos Sánchez Ruedo (Dueño de un pequeño negocio)}

    Carlos Sánchez es el propietario de un pequeño negocio local especializado en la venta de comida rápida. Inspirado por la exitosa estrategia de KFC de posicionarse como una marca influencer en las redes sociales decide implementar una estrategia similar para promocionar su negocio y aumentar su visibilidad en línea. Sin embargo, debido a las limitaciones de recursos y la falta de experiencia tecnológica no ha dado el paso todavía.

\section{Historias de usuario}

    \subsection{María Rodríguez Espinosa (Creadora de contenido)}

    \begin{enumerate}
        \item [HU01] Como María Rodríguez, creadora de contenido de memes para KFC, necesito poder editar memes y almacenarlos.
        \item [HU02] Como María Rodríguez, creadora de contenido de memes para KFC, siempre he tenido problemas encontrando memes antiguos que he creado.
    \end{enumerate}

    \subsection{Juan Pérez Ruiz (CTO)}

        \begin{enumerate}
            \item [HU02] Como Juan Pérez, CTO, necesito que puedan colaborar varios usuarios en la creación de los memes además de administrar quién del equipo puede ver, editar o eliminar los memes.
        \end{enumerate}

    \subsection{Carlos Sánchez Ruedo (Dueño de un pequeño negocio)}

        \begin{enumerate}
            \item [HU03] Como Carlos Sánchez, dueño de un pequeño negocio, necesito un sistema que se pueda usar desde un pequeño ordenador que tengo en la tienda.
            \item [HU04] Como Carlos Sánchez, dueño de un pequeño negocio, quiero que me facilite la compartición con las redes sociales, puesto que no sé mucho de ellas y no tengo a nadie en mi equipo de marketing. 
        \end{enumerate}

\section{Milestones}

Las descripciones de productos concretos o productos mínimos viables (MVP) son una herramienta útil para definir y comunicar el alcance de un proyecto. En este caso, se han identificado los siguientes milestones:

\begin{enumerate}
    \item [M00] - Documentación, planificación y configuración inicial
    \item [M01] - Domain Driven Design
    \item [M02] - Almacenamiento y acceso
    \item [M03] - Búsqueda y gestión
    \item [M04] - Tests y despliegue
    \item [M05] - Seguridad y privacidad
\end{enumerate}