\chapter{Introducción}

En una era digital donde las tendencias cambian en un abrir y cerrar de ojos, los memes se han convertido en una forma popular de comunicación que se mantiene atemporal y relevante cambiando y adaptándose por cada tendencia.

Los memes están tan integrados en nuestra cultura que los utiliza desde el estudiante que busca hacer sus presentaciones y exposiciones más memorables y llevaderas hasta una empresa que busca captar la atención de audiencia en las redes sociales.

La integración de los memes en el contenido no solo buscan vender el contenido de manera más fresca, entretenida y atractiva, sino que otras veces, solo se busca el entretenimiento y la diversión pura y simple de los consumidores.

Imagine una escena en la que un grupo de amigos se reúne y discuten sobre los últimos acontecimientos políticos cuando, de repente, un meme ingenioso y fuera de lugar se convierte en el centro de la conversación, rompiendo el hielo y generando risas. Imagine ahora una situación algo más seria en el que un jefe de proyecto está presentando un proyecto a un posible inversor y, para ilustrar un punto clave, muestra un meme que todos los presentes reconocen y entienden, generando una conexión instantánea y una sensación de confianza y cercanía.

En todos estos casos y en muchos más, los memes se han convertido un medio para conectar con el consumidor de una manera única y efectiva.

Sin embargo, a pesar de su uso generalizado, la gestión, centralización y organización de memes sigue siendo un desafío para muchos creadores de contenido, empresas y personas.

\section{Motivación}

Como hemos comentado la creación y compartición de memes es una actividad de lo más común por lo que la falta de un sistema centralizado limita su potencial para todo tipo de usuarios. Esta falta de una solución informática específica se traduce en una pérdida de tiempo y recursos, así como en una limitación en la capacidad de generar contenido relevante y oportuno. Lo cual es una característica esencial en el mundo de los memes.

La motivación detrás de este proyecto radica en ofrecer una solución que subsane las dificultades que los usuarios encuentran a la hora de gestionar estos medios de información. Estas dificultades van desde la dificultad de encontrar memes antiguos hasta la falta de herramientas colaborativas y centralizadas. Como se puede apreciar los problemas pueden llegar a ser muy diversos y afectar a diferentes tipos de usuarios.

El proyecto `Memes para todos` aspira a proporcionar una plataforma robusta y fácil de usar que simplifique el proceso de creación, almacenamiento y distribución de memes para todo tipo de usuarios, sin importar su experiencia previa o contexto.

\section{Definición del problema}

Dentro de este marco de trabajo, para realizar una mejor de definición del problema, se han identificado una serie de personas con necesidades y objetivos específicos que junto con las historias de usuario describen los problemas que se pretenden resolver con este proyecto.

\section{Objetivos Iniciales}

De las historias de usuario y personas identificadas se han extraído los siguientes objetivos iniciales:

\begin{enumerate}
    \item Diseñar una solución que sea lo más económica posible, permitiendo su instalación en servidores en la oficina o su despliegue en contenedores sin costos excesivos, lo que garantiza una mayor accesibilidad y viabilidad para organizaciones de todos los tamaños y presupuestos.
    \item Se deberá crear un sistema que comprenda una gama de clientes lo más amplia posible, ofreciendo una solución que sea capaz de adaptarse a las diferentes necesidades y preferencias de los usuarios.
    \item Se deberá garantizar la accesibilidad y facilidad de uso desarrollando una aplicación que se pueda utilizar incluso desde un ordenador de bajas prestaciones.
    \item La licencia del proyecto deberá de permitir su uso por cualquier organización o personas sin importar el fin con el que se utilice, ya sea para uso personal, educativo o empresarial.
\end{enumerate}