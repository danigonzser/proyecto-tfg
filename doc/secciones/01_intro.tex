\chapter{Introducción}

En la era digital contemporánea, la tecnología ha transformado radicalmente la forma en la que nos comunicamos, trabajamos, nos divertimos, etc. 
La tecnología ha hecho que la información esté disponible en cualquier momento y en cualquier sitio. A la vez que el consumo de contenido,
también ha aumentado la creación de contenido que ha alcanzado proporciones sin precedentes con personas de todas las partes del mundo contribuyendo a un vasto
océano de información compuesto por imágenes, vídeos, artículos, etc. Este fenómeno ha dado lugar a una nueva era de expresión y comunicación, donde la
generación y difusión de contenido digital se ha convertido en una parte integral de la vida cotidiana de las personas.

\section{Contexto y Motivación}

En este escenario de generación de contenido digital sin parangón se presenta un desafío para los creadores de contenido: la gestión y almacenamiento eficiente, eficaz
y ágil de dicho contenido. A pesar de contar con diversas plataformas de almacenamiento en la nube y soluciones de gestión de contenidos digitales, la mayoría de
estas soluciones tienen diferentes limitaciones que afectan a la organización, ordenación y accesibilidad de este contenido. La necesidad de una solución que permita
a los usuarios almacenar, organizar y acceder a su amplio repertorio contenido digital de forma eficiente, eficaz y ágil es evidente.

Esta falta de soluciones eficientes, eficaces y ágiles para la gestión y almacenamiento de contenido digital es la motivación principal de este proyecto. Aquí se
presenta una oportunidad significativa para desarrollar un sistema más robusto y adaptado a las necesidades actuales.

Al mejorar la eficiencia, eficacia y agilidad a la hora de la gestión y almacenamiento del contenido digital los usuarios pueden dedicarse a lo que realmente les importa:
la creación de contenido permitiéndoles así dedicarle más tiempo a la creatividad y a la expresión que requiere la creación de contenido digital sin miedo a perder o
no poder acceder a su contenido.

En este contexto y motivación, el presente proyecto se propone abordar estas limitaciones, diseñando y desarrollando un sistema de almacenamiento online que no solo almacene, 
sino que también organice, ordene y facilite el acceso a contenido digital de manera intuitiva y eficiente.

\section{Definición del problema}

Como se ha señalado previamente, la generación masiva de contenido digital ha creado un problema de gestión y almacenamiento del mismo. A pesar de la existencia de
soluciones en la nube y plataformas de gestión de contenido, existen limitaciones a la hora de la gestión y organización de imágenes. La ausencia de una solución que
permita la gestión y organización de imágenes de forma centralizada y eficiente es el problema que se pretende abordar con este proyecto.

    \subsection{Usuarios identificados}

    \begin{itemize}
        \item \textbf{Nombre}: María Rodríguez.
        \item \textbf{Edad}: 30 años.
        \item \textbf{Educación}: Licenciada en comunicación digital.
        \item \textbf{Ocupación}: Creadora de contenido.
        \item \textbf{Conocimientos y actitud hacia la tecnología}: Tiene conocimientos avanzados en herramientas digitales y plataformas de creación de contenido. Es entusiasta y proactiva hacia las nuevas tecnologías, siempre dispuesta a explorar nuevas soluciones que mejoren su experiencia y flujo de trabajo a la hora de la creación y gestión de contenido digital.
        \item \textbf{Dispositivos}:
            \begin{itemize}

            \item \textbf{¿Cuáles usa?}: Ordenador portátil de última generación, teléfono móvil de última generación y tablet.
            \item \textbf{¿Cuándo los usa?}: Durante todo el día.
            \item \textbf{¿Para qué los usa?}: Para mantenerse al tanto de las tendencias, interactuar con su audiencia y crear contenido.

            \end{itemize}
        \item \textbf{Misión}:
        \item 
            \begin{itemize}

                \item \textbf{¿Utilizaría nuestra aplicación?}: Sí.

                \item \textbf{¿Qué le gustaría que tuviera?}: Un sistema de almacenamiento en la nube que le permita subir y organizar fácilmente su contenido desde cualquier lugar. Le gustaría tener un acceso rápido y sencillo desde cualquier dispositivo con conexión a Internet, una interfaz intuitiva, un sistema de gestión de versiones, seguridad y privacidad, un sistema de búsqueda avanzada que permita encontrar cualquier contenido.

            \end{itemize}
        
    \end{itemize}

    \begin{itemize}
        \item \textbf{Nombre}: Juan Pérez.
        \item \textbf{Edad}: 40 años.
        \item \textbf{Educación}: Ingeniero de Sistemas.
        \item \textbf{Ocupación}: CTO.
        \item \textbf{Conocimientos y actitud hacia la tecnología}: Conocimientos avanzados en sistemas y tecnologías. Actitud positiva y pragmática.
        \item \textbf{Dispositivos}:
            \begin{itemize}
            \item \textbf{¿Cuáles usa?}: Ordenador de sobremesa y teléfono móvil. 
            \item \textbf{¿Cuándo los usa?}: Principalmente durante las horas laborales.
            \item \textbf{¿Para qué los usa?}: Para trabajar, informarse y comunicarse.
            \end{itemize}
        \item \textbf{Misión}:
            \begin{itemize}

                \item \textbf{¿Utilizaría nuestra aplicación?}: No. Prefiere un almacenamiento local, no considera necesario un sistema complejo y desconfía de la nube. Utiliza Microsoft OneDrive que ya le brinda la flexibilidad de acceso a los archivos, seguridad, privacidad y el sistema de búsqueda y gestión.

                \item \textbf{¿Qué le gustaría que tuviera?}: Le gustaría que no tuviera complicaciones a la hora de la instalación y configuración. Todas las facilidades que ya le brinda Microsoft OneDrive más alguna característica única e interesante para él.

            \end{itemize}
        
    \end{itemize}

    \begin{itemize}
        \item \textbf{Nombre}: Laura Martínez.
        \item \textbf{Edad}: 35 años.
        \item \textbf{Educación}: Licenciada en Bellas Artes.
        \item \textbf{Ocupación}: Fotógrafa profesional.
        \item \textbf{Conocimientos y actitud hacia la tecnología}: Tiene conocimientos avanzados en técnicas fotográficas y edición de imágenes. Prefiere soluciones sencillas y fáciles de usar. No le gusta perder el tiempo en configuraciones y opciones complejas.
        \item \textbf{Dispositivos}:
            \begin{itemize}

            \item \textbf{¿Cuáles usa?}: Ordenador portátil de alta gama y dispositivo móvil inteligente.
            \item \textbf{¿Cuándo los usa?}: Durante todo el día.
            \item \textbf{¿Para qué los usa?}: Revisar correos, redes sociales además de editar y visualizar imágenes.

            \end{itemize}
        \item \textbf{Misión}:
        \item 
            \begin{itemize}

                \item \textbf{¿Utilizaría nuestra aplicación?}: No. Siempre ha experimentado una pérdida de calidad al subir su contenido a la nube. Prefiere tener un almacenamiento local como el NAS que tiene para almacenar y organizar su contenido.

                \item \textbf{¿Qué le gustaría que tuviera?}: Su sistema idea es un sistema que facilite la organización, gestión y almacenamiento de su contenido de forma intuitiva, simple y eficaz sin pérdidas de calidad.

            \end{itemize}
        
    \end{itemize}

    \begin{itemize}
        \item \textbf{Nombre}: Carlos Sánchez
        \item \textbf{Edad}: 45 años.
        \item \textbf{Educación}: Licenciado en administración de empresas.
        \item \textbf{Ocupación}: Dueño de un pequeño negocio.
        \item \textbf{Conocimientos y actitud hacia la tecnología}: Conocimientos muy básicos en tecnología. Utiliza la tecnología para ganar difusión en su negocio aunque no sea un experto. Actitud conservadora y pragmática.
        \item \textbf{Dispositivos}:
            \begin{itemize}

            \item \textbf{¿Cuáles usa?}: Ordenador de sobremesa y teléfono inteligente.
            \item \textbf{¿Cuándo los usa?}: Durante las horas laborales.
            \item \textbf{¿Para qué los usa?}: Para atender llamadas y mensajes de clientes, comprobar el correo, redes sociales y gestionar su negocio.

            \end{itemize}
        \item \textbf{Misión}:
        \item 
            \begin{itemize}

                \item \textbf{¿Utilizaría nuestra aplicación?}: Sí, pero al tener una conexión inestable (a veces no funciona Internet) duda de si es una buena solución para él.

                \item \textbf{¿Qué le gustaría que tuviera?}: Valoraría mucho que el sistema ofreciera opciones de gestión offline y que no dependa de tener una conexión estable a Internet además de que fuera fácil de implementar y usar.

            \end{itemize}
        
    \end{itemize}

    \begin{itemize}
        \item \textbf{Nombre}: Marta González.
        \item \textbf{Edad}: 28 años.
        \item \textbf{Educación}: Graduada en Diseño Multimedia.
        \item \textbf{Ocupación}: Diseñadora Autónoma.
        \item \textbf{Conocimientos y actitud hacia la tecnología}: Tiene conocimientos avanzados en software de diseño gráfico y multimedia. Tiene una actitud positiva hacia la tecnología para mejorar su eficiencia en la organización y archivos digitales.
        \item \textbf{Dispositivos}:
            \begin{itemize}

            \item \textbf{¿Cuáles usa?}: Ordenador de sobremesa potente, teléfono inteligente y tableta gráfica.
            \item \textbf{¿Cuándo los usa?}: Durante las horas diurnas pero también algunas veces por la noche.
            \item \textbf{¿Para qué los usa?}: Para realizar sus proyectos de ilustración y diseño.

            \end{itemize}
        \item \textbf{Misión}:
        \item 
            \begin{itemize}

                \item \textbf{¿Utilizaría nuestra aplicación?}: Sí. Aunque siempre que ha utilizado un sistema para la organización de contenido ha tenido problemas cuando la cantidad del mismo es muy grande.

                \item \textbf{¿Qué le gustaría que tuviera?}: Le gustaría que fuera realmente escalable y mantenible aun con cantidades grandes de contenido.

            \end{itemize}
        
    \end{itemize}

\section{Objetivos}

La esencia de los objetivos radica en diseñar y desarrollar un sistema de almacenamiento en la nube. Este sistema se enfoca en resolver los diferentes problemas
identificados durante el análisis del problema. 

La experiencia de usuario es un pilar fundamental de nuestro proyecto. Varios de los problemas y de las personas identificadas en el análisis del problema
pueden resolverse con una útil, accesible, intuitiva y simple interfaz de usuario reduciendo los clics necesarios y disminuyendo el tiempo de carga,
todo con el objetivo de aumentar la satisfacción del usuario.

La gestión, ordenación y organización del contenido digital del usuario debe ser nuestro principal objetivo. Todo esto para reducir el tiempo que el usuario
dedica a la gestión de su contenido y aumentar el tiempo que dedica a la creación de contenido.

La compatibilidad con diferentes dispositivos y sistemas operativos es otro de los objetivos de este proyecto. El usuario debe poder acceder a su contenido
desde cualquier dispositivo con conexión a Internet y con cualquier sistema operativo.

La seguridad y privacidad del contenido del usuario es otro de los objetivos de este proyecto. El usuario debe poder confiar en el sistema de almacenamiento
y no debe preocuparse por la seguridad y privacidad de su contenido.

Además de estos objetivos existen otros como la incorporación de gestión de versiones para mejorar la experiencia del usuario y la implementación de un sistema
de búsqueda avanzada que permita al usuario encontrar cualquier contenido de forma rápida y sencilla.

Finalmente, otro objetivo de este proyecto es relativo a la evaluación y validación del sistema por parte del cliente, que, en nuestro caso es el tutor
y el tribunal. El sistema debe ser evaluado y validado por el cliente para comprobar que cumple con los requisitos y objetivos establecidos.