\chapter{Introducción}

Los memes se han convertido en un lenguaje universal en el mundo digital, traspasando fronteras, culturas y generaciones. Desde las conversaciones informales en grupos de chat, hasta en las estrategias de marketing de grandes empresas, los memes son herramientas poderosas para transmitir ideas, emociones y mensajes de forma rápida y efectiva. Sin embargo, a pesar de su uso generalizado, la gestión, centralización y organización de memes sigue siendo un desafío para muchos creadores de contenido, empresas y personas.

\section{Motivación}

Crear y compartir memes es una actividad común, la falta de un sistema centralizado limita su potencial tanto para creadores individuales como para empresas y organizaciones. Esta falta de una solución informática específica se traduce en una pérdida de tiempo y recursos, así como en una limitación en la capacidad de generar contenido relevante y oportuno.

La motivación detrás de este proyecto radica en ofrecer una solución que resuelva problemas y satisfaga las necesidades de diversos usuarios en diferentes contextos. El proyecto `Memes para todos` aspira a proporcionar una plataforma robusta y fácil de usar que simplifique el proceso de creación, almacenamiento y distribución de memes.

\section{Definición del problema}

Dentro de este marco de trabajo, para realizar una mejor de definición del problema, se han identificado una serie de personas con necesidades y objetivos específicos que junto con las historias de usuario describen los problemas que se pretenden resolver con este proyecto.

\section{Milestones}

Las descripciones de productos concretos o productos mínimos viables (MVP) son una herramienta útil para definir y comunicar el alcance de un proyecto. En este caso, se han identificado los siguientes milestones:

\begin{enumerate}
    \item [M00] - Documentación, planificación y configuración inicial
    \item [M01] - Domain Driven Design
    \item [M02] - Almacenamiento y acceso
    \item [M03] - Búsqueda y gestión
    \item [M04] - Tests y despliegue
    \item [M05] - Seguridad y privacidad
\end{enumerate}