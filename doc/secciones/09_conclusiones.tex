\chapter{Conclusiones y trabajos futuros}

Finalmente, en este capítulo se presentarán las conclusiones obtenidas tras la realización del proyecto y se propondrán posibles trabajos futuros que se podrían llevar a cabo para mejorar la aplicación.

Las historias de usuario que se marcaron al principio del proyecto se han cumplido en su totalidad, aunque una de ellas no se ha podido implementar al completo. Así pues, se va a proceder a enumerar las historias de usuario y a describir su estado actual:

\begin{itemize}
  \item[\bien] \textbf{HU01 Como Cristina Contreras Márquez, estudiante de Ingeniería Informática, como hago memes mientras voy haciendo las presentaciones quiero rapidez y sencillez por lo que no quiero tener que hacer el meme desde cero:} Esta historia de usuario se ha satisfecho completamente. Cristina Contreras Márquez puede acceder a algún catálogo hecho con memes realizados por otro usuario y modificarlo a su gusto.
  \item[\bien] \textbf{HU02 Como Cristina Contreras Márquez, estudiante de Ingeniería Informática, quiero crear memes que puedan ser modificados o reutilizados por otros usuarios.} De igual manera, Cristina Contreras Márquez puede elaborar memes y subirlos a la plataforma para que otros usuarios puedan modificarlos.
  \item[\bien] \textbf{HU03 Como Ted Johnson González, experto conferenciante, quiero acceso rápido y sencillo a memes relevantes y atractivos para incluir en mis presentaciones sin necesidad de confeccionarlos yo mismo.} En este caso, Ted Johnson González puede acceder a catálogos de memes que otros usuarios han elaborado y puede descargarlos para incluirlos en sus presentaciones.
  \item[\regular] \textbf{HU04 Como departamento de marketing, queremos colaborar en un meme varios miembros del equipo.} Por último, tenemos al departamento de marketing cuya historia de usuario no se ha cumplido al completo dado que no se ha implementado la funcionalidad de permisos, roles y colaboración en la aplicación, pero esto no impide que varios miembros del equipo puedan colaborar en un meme.
\end{itemize}

Por lo que acabamos de ver se puede afirmar que el proyecto ha sido un éxito en cuanto a la implementación de las historias de usuario marcadas al principio del proyecto. A continuación se presentarán las conclusiones obtenidas tras la realización del proyecto.

\section{Conclusiones}

Tras haber trabajado en este proyecto con últimas tecnologías como es Next.js 15 que salió este pasado 21 de octubre de 2024, he llegado a la conclusión que este marco de desarrollo web de React otorga al programador una experiencia de desarrollo muy agradable, ágil y eficiente. Además, la casi cero configuración que requiere el despliegue de la aplicación en Vercel cierra el flujo de trabajo de cualquier proyecto software de manera muy satisfactoria.

En cuanto a la otra tecnología ‘troncal’ utilizada como es Fabric.js 5.0, he de decir que me ha sorprendido gratamente la facilidad con la que se pueden crear lienzos interactivos y la cantidad de funcionalidades que ofrece. Al principio, cuesta bastante acostumbrarse a esta nueva forma de trabajar con elementos gráficos, pero es una librería muy potente y versátil.

A título personal siempre he querido experimentar el desarrollo de una aplicación web no solo con las últimas tecnologías sino también con enfoques ágiles y debo decir que ha sido una experiencia muy enriquecedora y que desde luego voy a poder aplicar tanto en el ámbito profesional como en futuros proyectos personales.

\section{Trabajos futuros}

Como se ha mencionado en la introducción de este capítulo la última historia de usuario (HU04) no se ha implementado al completo. Por lo que se propone como trabajo futuro la implementación de la funcionalidad de permisos, roles y colaboración en la aplicación. De esta manera, varios miembros del equipo podrán colaborar en un meme y se podrán establecer roles de usuario como administrador, editor, colaborador, etc.

Como se ha hecho durante toda la planificación de la aplicación se va a establecer un quinto hito o milestone contemplando todos esos trabajos futuros.

\subsection{Milestone 5: Autenticación, colaboración y roles}

El PMV u objetivo de este milestone es la implementación de los roles de usuario respecto a la modificación de memes, su respectiva lógica de colaboración y permisos así como el sistema de autenticación y perfiles.

Algunos de los objetivos que se pretenden alcanzar en este milestone son:

\begin{itemize}
  \item \textbf{Roles de usuario.}
  \item \textbf{Lógica de colaboración y permisos.}
  \item \textbf{Autenticación.}
  \item \textbf{Perfiles.}
\end{itemize}

