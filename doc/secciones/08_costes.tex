\chapter{Costes}

En este capítulo se detallarán los costes asociados al desarrollo del proyecto, incluyendo los costes de personal, hardware y software. Además, se describirán los costes indirectos y se presentará un análisis de los costes totales del proyecto. Este proyecto \textbf{ha durado un total de 9 meses}. Las cuotas de amortización se han calculado basándonos en un período de 4 años y cuyos porcentajes se han sacado de la tabla de amortizaciones simplificadas de la Agencia Tributaria Española~\cite{agencia2023manual}.

\section{Costes de Hardware}

Para poder llevar a cabo este proyecto se ha necesitado del siguiente hardware:

\begin{table}[H]
  \centering
  \begin{tabular}{|l|l|l|l|l|}
    \hline
    \textbf{Hardware}   & \textbf{Valor} & \textbf{Amortización (\%)} & \textbf{Tiempo de uso} & \textbf{Amortización} \\ \hline
    HP Pavilion ck004ns & 1000 €         & 25\% a 4 años              & 9 meses                & Total: 187,5 €        \\ \hline
  \end{tabular}
  \caption{Tabla de los costes relativos del Hardware al desarrollo.}
  \label{table:1}
\end{table}

\section{Costes de Software de desarrollo}

Para el desarrollo del proyecto no se han utilizado herramientas de pago, por lo que no se han incurrido en costes de software.

\section{Costes de personal}

\begin{table}[H]
  \centering
  \begin{tabular}{|l|l|l|l|}
    \hline
    \textbf{Hardware} & \textbf{Sueldo} & \textbf{Tiempo proyecto} & \textbf{Total}   \\ \hline
    Sueldo medio *    & 21.509€ (año)   & 9 meses                  & Total: 16.131,75 \\ \hline
  \end{tabular}
  \caption{Tabla de los costes del personal relativos al desarrollo.}
  \label{table:1}
\end{table}

* El sueldo medio se ha calculado a partir de la \href{https://www.glassdoor.es/Sueldos/desarrollador-full-stack-junior-sueldo-SRCH_KO0,31.htm}{media de los sueldos de los desarrolladores full stack junior en España}.

\section{Costes de despliegue}

Para el despliegue de la aplicación, como se ha mencionado en el capítulo de implementación, se ha empleado Vercel que dispone de un plan gratuito. Este plan es de sobra suficiente para el despliegue de la aplicación, por lo que no se han incurrido en costes de despliegue.

Aunque sea idea para proyectos personales y pequeñas aplicaciones existen diferentes limitaciones en este plan como el limitado tiempo de construcción a 1 hora al día junto con un ancho de banda de hasta 100 GB al mes.

\subsection{Costes de la persistencia de datos}

Para el uso de la base de datos de \textbf{Vercel Postgres} los recursos que nos ofrece el plan gratuito son suficiente, pero también debemos ser conscientes de las limitaciones del mismo. Por ejemplo, el tiempo de computación es de 60 horas mensuales y el almacenamiento de 256 MB lo cual puede llegar a ser poco incluso para el proyecto actual pero con la posibilidad de escalar a un plan de pago en caso de ser necesario.

\subsubsection{Estimación de capacidad con el plan gratuito}

Actualmente, el plan gratuito de Vercel Postgres proporciona hasta 256 mB de almacenamiento. Para valorar la capacidad que ofrece este plan, se ha estimado el tamaño promedio de cada meme, considerando que el proyecto almacenará memes en formato JSON. El tamaño medio de un meme vamos a establecer en 500 kB.

Con esta cifra, el almacenamiento gratuito soportaría aproximadamente 512 memes. Esta capacidad es suficiente para el desarrollo y prueba de la aplicación, así como para entornos de producción de pequeña escala o con necesidades de almacenamiento limitadas.

\subsection{Costes estimados por uso}

En el caso de que el proyecto crezca y se necesite más capacidad de almacenamiento, Vercel ofrece a sus usuarios el poder exceder los límites del plan gratuito mediante la contratación de un plan de pago. Una vez superado el límite, pagará por el uso de la misma. En nuestro caso por cada gB adicional se añadirán 0.10 € al mes.
