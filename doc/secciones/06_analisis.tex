\chapter{Análisis del problema}

En este capítulo, se explicará la metodología seguida y se presentará un análisis de requisitos. Comenzaremos entendiendo el contexto del problema e identificando los requisitos a través de las historias de usuario y los usuarios identificados previamente. Posteriormente, se elaborará el código generado del DDD que modele las estructuras de datos necesarias para definir el problema.

Para garantizar una estructura coherente, organizada, eficiente y alineada con las mejores prácticas de desarrollo de software, se ha optado por un enfoque ágil junto con \textit{domain driven design} (DDD). Este enfoque permite profundizar en la comprensión y modelado del problema, a la vez el desarrollo se hace de forma incremental, continua y adaptativa.

\section{Domain-Driven Design}

En este capítulo, se hará un análisis detallado de los requisitos del proyecto haciendo uso del Diseño Orientado al Dominio o \textit{domain-driven design} que para mayor rapidez y concisión denominaremos como \textit{DDD}. Esta filosofía de desarrollo de software nos ayudará a comprender y modelar el dominio empresarial, asegurando que nuestras soluciones se alineen con las necesidades específicas de los usuarios y del negocio.

El Diseño Orientado al Dominio (DDD) es una filosofía de desarrollo de software que se centra en comprender y modelar el dominio empresarial para alinear el software con las necesidades del negocio. Introducido por Eric Evans en su libro~\cite{evans2004domain}, DDD ha evolucionado con aportes de numerosos profesionales.

En esencia, \textit{DDD} aborda la complejidad enfocándose en el ``dominio'', el contexto específico del software. Promueve el uso de un ``lenguaje ubicuo'', un lenguaje común entre desarrolladores y partes interesadas, para garantizar que el software refleje con precisión el ámbito empresarial. La modelización en \textit{DDD} no busca crear el modelo más realista, sino uno útil y apropiado para su propósito, similar a cómo una película representa la realidad de manera selectiva para cumplir su objetivo.

Hemos elegido \textit{DDD} porque proporciona una estructura clara y coherente para manejar la complejidad del dominio del proyecto, asegurando que las soluciones técnicas se alineen estrechamente con las necesidades del negocio y los usuarios. Esta metodología nos permitirá identificar y definir las entidades principales que se necesitan para satisfacer los requisitos del sistema, facilitando la comunicación y la comprensión entre todas las partes involucradas.

Hoy en día, DDD sigue siendo ampliamente utilizado en diversos proyectos, con numerosos casos de éxito en la industria que demuestran su eficacia, como se detalla en esta \href{https://blog.bitsrc.io/demystifying-domain-driven-design-ddd-in-modern-software-architecture-b57e27c210f7}{publicación}. Empresas como Netflix, Uber y Airbnb han adoptado DDD con notable éxito.

\subsection{Enfoque Ágil y DDD}

Es importante destacar que el diseño impulsado por el dominio (Domain-Driven Design, DDD) en~\cite{evans2004domain} no abandona la filosofía ágil. Aunque este libro no está vinculado a una metodología en particular, se orienta hacia la nueva familia de ``procesos de desarrollo ágil''. Específicamente, asume que se emplean dos prácticas en el proyecto, las cuales son prerrequisitos para aplicar el enfoque presentado en este libro.

\begin{itemize}
    \item \textbf{El desarrollo es iterativo.}
    \item \textbf{Relación estrecha entre quiénes conocen el dominio y quiénes saben cómo construir \textit{software}.}
\end{itemize}

\section{Análisis de requisitos}

Los requisitos han sido identificados a partir de las historias de usuario~\ref{sec:historias_de_usuario} anteriormente definidas y los usuarios identificados en~\ref{sec:usuarios_identificados}. Estos requisitos son los que llamamos los requisitos básicos o esenciales a partir de los cuales se desarrollará el diseño.

\begin{itemize}
    \item \textbf{Se debe proporcionar un editor de memes.}
    \begin{itemize}
        \item[-] Historias de usuario requeridas: HU01, HU04 y HU02.
    \end{itemize}
    \item \textbf{Se debe proporcionar un catálogo de memes predefinidos.}
    \begin{itemize}
        \item[-] Historias de usuario requeridas: HU01, HU02 y HU03.
    \end{itemize}
    \item \textbf{La solución debe poder permitir colaboración en la creación de memes.}
    \begin{itemize}
        \item[-] Historias de usuario requeridas: HU04.
    \end{itemize}
    \item \textbf{Se debe proporcionar un sistema de búsqueda avanzada de memes.}
    \begin{itemize}
        \item[-] Historias de usuario requeridas: HU03.
    \end{itemize} 
\end{itemize}