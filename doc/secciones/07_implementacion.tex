\chapter{Implementación}

Una vez explicada la metodología, planificación y análisis, se va a proceder con explicación de la implementación de la solución. Así como en las otras secciones, se van a explicar todas y cada una de las decisiones tomadas, problemas encontrados y soluciones propuestas.

\section{Análisis de las posibles soluciones}

En este apartado, se analizarán las posibles soluciones para el desarrollo de la aplicación.

\subsection{Desarrollo de una \href{https://apps.nextcloud.com/}{aplicación} para \href{https://nextcloud.com/es/}{\textit{Nextcloud}}}

\textit{Nextcloud} es una plataforma de software libre que permite a los usuarios almacenar y sincronizar archivos, contactos, calendarios y más. Es una excelente opción para el desarrollo de nuestra aplicación, ya que proporciona una base sólida y una amplia gama de funcionalidades que pueden ser aprovechadas para implementar la funcionalidad de creación y gestión de memes. Para el desarrollo de una app en el ecosistema de \textit{Nextcloud}, se utiliza PHP y Vuejs. Además, \textit{Nextcloud} proporciona una API robusta que permite a los desarrolladores acceder a las funcionalidades del sistema.

Esta solución tiene varios problemas para el usuario o cliente final: 

\begin{itemize}
    \item \textbf{Complejidad de la instalación:} La instalación de \textit{Nextcloud} puede ser complicada para los usuarios no técnicos. Por ejemplo para Ted Johnson González (experto conferenciante) que no tienen experiencia ni conocimientos técnicos podría ser un problema.
    \item \textbf{Consumo de recursos:} \textit{Nextcloud} es una aplicación bastante exigente en términos de recursos, especialmente si se gestiona una gran cantidad de datos o usuarios como podría ser el caso del departamento de marketing de Corporate Solutions. Desarrollando nuestra propia aplicación, podemos optimizar los recursos para que los requisitos sean mínimos.
\end{itemize}

Por otro lado, la implementación de base de funcionalidades como la sincronización en local y en remoto, el almacenamiento y la búsqueda de archivos y la gestión de usuarios no justifica el tiempo que se va a dedicar a adaptarse al \href{https://nextcloud.com/developer/}{entorno de desarrollo} que proporciona Nextcloud para desarrollar apps, ya que estas funcionalidades se pueden aplicar mediante la implementación de repositorios o desde cero con un coste de tiempo y recursos mucho menor. 

\subsection{Desarrollo de una app móvil}

Otra posible solución sería el desarrollo de una aplicación móvil que permita a los usuarios crear y gestionar memes desde sus dispositivos móviles. Esta solución tiene varias ventajas, como la facilidad de uso y la accesibilidad, ya que los usuarios pueden elaborar memes en cualquier momento y lugar. Podría ser de bastante utilidad para todos y cada uno de los usuarios identificados.

Sin embargo, esta solución también tiene sus inconvenientes. Por ejemplo, el desarrollo de aplicaciones móviles limita bastante la audiencia a la que se puede llegar, puesto que no todos los usuarios tienen un dispositivo móvil o prefieren utilizar un ordenador. La idea de este proyecto es justamente el poder llegar a un público amplio y diverso, por lo que esta solución no sería la más adecuada.

\subsection{Desarrollo de una aplicación de escritorio}

En esta solución que se ha considerado es el desarrollo de una aplicación de escritorio que permita a los usuarios generar y gestionar memes desde sus ordenadores. Esta solución tiene varias ventajas como la facilidad que va a tener el usuario final para adaptarse a este tipo de entorno, pues es donde se realizan este tipo de tareas para que la edición pueda llegar a ser más compleja.

De nuevo, esta solución limita bastante la audiencia a la que se puede llegar, ya que no todos los usuarios tienen un ordenador o prefieren un dispositivo móvil (la cual es la mayoría actual).

\subsection{Desarrollo de una aplicación web}

El desarrollo de una aplicación web tiene lo bueno de las dos soluciones anteriores. Permite al usuario conectarse desde cualquier dispositivo sea PC o móvil. El desarrollo web es el más popular en la actualidad por lo que es bastante fácil encontrar comunidades, preguntas, documentación, artículos, etc.

La idea detrás de esta solución reside en una aplicación web que sea igualmente accesible desde cualquier dispositivo con conexión y un navegador adecuado. Una vez accedes a la plataforma web, puedes crear, editar, almacenar, compartir y buscar memes. La forma de almacenar los memes podría llegar a hacerse en local o en remoto.

\section{Selección de lenguajes de programación o tecnologías}

El proyecto se va a desarrollar como una aplicación web teniendo esta usualmente una parte del cliente o frontend, una parte del servidor, api o backend y una parte de persistencia o base de datos. La elección de la tecnología en sendas partes es crucial a la hora de la experiencia de desarrollo, compatibilidad, escalabilidad, mantenimiento, etc.

\subsection{Cliente o frontend}

Esta parte del proyecto es la encargada con la interacción directa con el usuario y debe ser lo más amigable, rápida, intuitiva y accesible posible. Los frameworks de frontend son herramientas esenciales para los desarrolladores web, ya que proporcionan módulos de código reutilizables, tecnologías estandarizadas y bloques de interfaz prefabricados. Estos elementos simplifican el desarrollo de aplicaciones e interfaces de usuario, eliminando la necesidad de codificar cada función u objeto desde cero.

En la actualidad, los frameworks inicialmente de frontend se han convertido en herramientas completas para el desarrollo de aplicaciones web o \textit{full stack web frameworks}. Algunos de los más populares son \textit{\href{https://es.react.dev/}{React}}, \textit{\href{https://angular.dev/}{Angular}}, \textit{\href{https://vuejs.org/}{Vue.js}}, \textit{\href{https://astro.build/}{Astro}} y textit{\href{https://nextjs.org/}{Next.js}}, siendo esta la alternativa que más está dando qué hablar recientemente. Cada uno de ellos tiene sus propias características, ventajas y desventajas, por lo que es importante evaluarlos cuidadosamente antes de tomar una decisión.

\section{Selección de la tecnología}

Anteriormente, se ha realizado la selección del lenguaje de programación que se iba a utilizar para desarrollar código para el DDD, el cual es TypeScript. Ahora, el siguiente paso es seleccionar la tecnología que va a ser empleada junto con TypeScript para desarrollar la solución.

Actualmente, cuando nos referimos a una tecnología web nos referimos a un conjunto de herramientas o framework que nos permiten desarrollar de forma más rápida y eficiente ayudándonos con ciertas características ya implementadas de base como la gestión de rutas, la gestión de estados, la gestión de peticiones, etc.

Una vez aclarado a qué nos referimos por tecnología, vamos a proceder al proceso en sí conforme se ha ido haciendo anteriormente, es decir, pasando primero por los criterios de búsqueda, selección, propuesta y finalmente, las conclusiones y justificación.

\subsection{Criterios de búsqueda}

Los criterios de búsqueda que se han fijado son:

\begin{itemize}
  \item \textbf{Typescript}: el lenguaje de programación que se va a usar es TypeScript, por lo que la tecnología seleccionada debe ser compatible con TypeScript.
  \item La tecnología debe ser compatible con \textbf{React}, una de las bibliotecas de JavaScript más populares en la actualidad para el desarrollo de interfaces de usuario, y que también es compatible con TypeScript. Su selección como criterio de búsqueda se basa en la escasa deuda técnica que presenta, su rapidez de uso y su estatus como un estándar, demostrado su aplicación en entornos reales. Además, cuenta con una amplia cantidad de recursos, bibliotecas adicionales y un mantenimiento continuo y estable.
\end{itemize}

\subsection{Criterios de selección}

Las propuestas encontradas tras los criterios de búsqueda serán evaluadas según:

\begin{itemize}
  \item Reactividad: la tecnología debe ser reactiva, es decir, debe ser capaz de reaccionar a los cambios de estado de la aplicación de forma eficiente.
\end{itemize}

\subsection{Propuestas}

\subsubsection{React}

\textit{React} es una biblioteca de JavaScript de código abierto para construir interfaces de usuario o componentes de la interfaz de usuario, desarrollada por Facebook y lanzada en 2013. Es una de las bibliotecas más populares y ampliamente utilizadas en la actualidad. React está diseñada para ayudar a los desarrolladores a crear aplicaciones con datos dinámicos, siendo simple, declarativa y fácil de integrar. Se centra exclusivamente en la interfaz de usuario, actuando como la Vista en los patrones de diseño. Además, puede integrarse con extensiones basadas en React que gestionan las partes no relacionadas con la interfaz de usuario de una aplicación web~\cite{react-wikipedia}. 

\begin{itemize}
  \item[\bien] En cuanto a reactividad es una de las mejores opciones y más empleadas desde siempre. Fue una de las primeras tecnologías que introdujo la reactividad como característica principal a la hora del diseño de interfaces de usuario.
\end{itemize}

Lo único en lo que podríamos ver un inconveniente es en que no es un framework completo, sino una librería, por lo que se necesitará de otras librerías para completar la aplicación, de aquí pueden surgir varios problemas como dependencias no actualizadas, incompatibilidades, falta de optimización, se añade peso a la aplicación, etc. Además, cuando se trata de aplicaciones grandes, React puede ser un poco más complicado de manejar que otros. Esto mismo puede generar cierta deuda técnica en el futuro al tener que integrar herramientas que ayuden a la hora de gestionar el estado.

\subsubsection{\textit{Next.js}}

\textit{Next.js} es un marco web de desarrollo frontend de React de código abierto creado por Vercel y Guillermo Rauch en 2016 que permite a los desarrolladores la renderización del lado del servidor y la generación de sitios web estáticos. \textit{Next.js} es una de las mejores opciones para el desarrollo de aplicaciones web modernas. Es el framework más popular y ampliamente utilizados en la actualidad~\cite{nextjs-wikipedia}.

Una de las mejores características que posee \textit{Next.js} es su soporte para el SEO que es esencial si quieres que tu aplicación sea indexada por los motores de búsqueda. Tiene soporte SSR, pero no lo hace de forma predeterminada. Por último, tiene integradas las rutas dinámicas, la pre-generación de páginas (renderizado prefijado) y la exportación estática. Estas funcionalidades son muy importantes a la hora de la construcción de aplicaciones web. Los casos de éxito son numerosos como TikTok, Notion, ChatGPT, Spotify, etc.

\begin{itemize}
  \item[\bien] Cuando hablamos de reactividad en una aplicación web, \textit{Next.js} ofrece muchas mejoras en este sentido aportando numerosas opciones de renderizado y optimización aportando características y componentes que lo facilitan en gran medida. Algunos de estos ejemplos son el componente Image, el componente Link, etc.
\end{itemize}

La incorporación de esas características únicas y diferenciadoras de \textit{Next.js} puede ser un problema a la hora de la escalabilidad de la aplicación, ya que puede ser que no se necesiten todas las características que ofrece y que se añada peso y lentitud a la aplicación. Además, la curva de aprendizaje puede ser un poco más pronunciada que con otras tecnologías.

\subsubsection{Astro}

Astro es un nuevo marco de desarrollo de aplicaciones web de código abierto que permite a los desarrolladores crear sitios web rápidos y eficientes. Fue desarrollado por Fred K. Schott y lanzado en 2022. Astro se centra en la creación de sitios web estáticos y dinámicos, proporcionando una estructura para el desarrollo de aplicaciones web modernas. Es conocido por su simplicidad y facilidad de uso~\cite{astro-wikipedia}.

Astro se ha convertido en una de las mejores opciones para el desarrollo de aplicaciones web modernas. Esta tecnología emergente ha ganado popularidad en los últimos años, principalmente porque, de forma predeterminada, no añade código JavaScript a menos que sea necesario. Otra de sus características más destacadas es su agnosticismo hacia la interfaz de usuario, lo que lo hace compatible con bibliotecas como React, Preact, Svelte, Vue, entre otras. Además, dado que el lenguaje de la interfaz de usuario es un superconjunto de HTML (la extensión Astro), la experiencia de desarrollo resulta familiar y accesible.

\begin{itemize}
  \item[\regular] En cuanto a la reactividad, Astro, al igual que \textit{Next.js}, ofrece muchas mejoras en este sentido aportando numerosas opciones de renderizado y optimización. Sin embargo, es una tecnología que se centra más en la elaboración de una aplicación web estática. 
\end{itemize}

Como se ha mencionado, Astro se centra más en la elaboración de una aplicación web estática, por lo que puede ser que no se adapte del todo a las necesidades de la aplicación.

\subsection{Conclusión}

Tras valorar cada uno de los criterios de selección de las diferentes tecnologías, se ha llegado a la conclusión de que la que mejor se adapta a las necesidades del proyecto y que más va a facilitar el desarrollo de la solución es \textbf{\textit{Next.js}}. Este marco de desarrollo web ofrece numerosas características y funcionalidades que influirán de forma positiva a lo largo de todo el proyecto. A mayores, el hecho de ser un framework por encima de una librería como React, va a facilitar la implementación de casi cualquier característica.