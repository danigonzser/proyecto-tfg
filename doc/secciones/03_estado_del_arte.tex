\chapter{Estado del arte}

En este capítulo se revisarán las herramientas y plataformas actuales que abordan funcionalidades relacionadas con la gestión, almacenamiento y edición colaborativa de memes. Esta revisión permitirá identificar los valores y limitaciones de las soluciones existentes frente a la propuesta de este proyecto.

Vamos a dividir este apartado en dos secciones: la primera sección que se centrará en la comparación con plataformas de creación y edición de memes y la segunda sección que se centrará en la comparación con plataformas de almacenamiento de contenido gráfico.

\section{Herramientas para la Creación y Edición de Memes}

Actualmente, existen diversas aplicaciones enfocadas en la creación y edición de memes, cada una con características específicas en cuanto a funcionalidad, facilidad de uso y accesibilidad. A continuación, se describen algunas de las herramientas más populares y sus características:

\begin{itemize}
  \item \textbf{Generación y creación de memes:} aplicaciones como \textit{\textbf{\href{https://imgflip.com/memegenerator}{Meme Generator de imgflip}}} y \textit{\textbf{\href{https://www.mematic.com/}{Mematic}}} permiten a los usuarios crear memes de forma rápida utilizando plantillas populares y ofreciendo opciones básicas de edición, como agregar texto y recortar imágenes. Sin embargo, estas aplicaciones presentan limitaciones cuando se requiere de una personalización más avanzada o colaboración. Además, la falta de opciones de almacenamiento organizado reduce su utilidad para quienes desean reutilizar y categorizar memes.
  \item \textbf{Aplicaciones orientadas a la edición online:} Plataformas como \href{https://www.canva.com/es_es/}{Canva} y \href{https://www.photopea.com/}{Photopea} permiten la edición de imágenes con mayores posibilidades de personalización y funciones avanzadas de edición gráfica. No obstante, estas herramientas suelen tener una curva de aprendizaje más alta, ya que están diseñadas para una gama más amplia de trabajos de diseño gráfico. A mayores, la colaboración y la organización de memes en colecciones no son características comunes en estas plataformas.
  \item \textbf{Aplicaciones con enfoque social:} redes sociales como \href{https://www.instagram.com/}{Instagram} permiten a los usuarios generar contenido visual con opciones para añadir texto y stickers, lo cual puede ser utilizado para crear memes. Sin embargo, estas plataformas no están diseñadas para la gestión centralizada de memes y carecen de una estructura adecuada para la categorización o reutilización de contenido, limitando su efectividad para un uso colaborativo o profesional.
\end{itemize}

\section{Plataformas de almacenamiento y organización de contenido visual}

La organización y almacenamiento centralizado es un aspecto clave para este proyecto, especialmente para usuarios que producen contenido de forma recurrente. En el mercado actual existen plataformas de almacenamiento de imágenes, pero ninguna está específicamente orientada a la gestión y reutilización de memes. Las plataformas destacadas incluyen:

\begin{itemize}
  \item \textbf{\href{https://www.google.com/intl/es_es/photos/about/}{Google Photos}} Esta plataforma de almacenamiento en la nube permite a los usuarios organizar sus archivos visuales mediante etiquetas y carpetas. A pesar de su popularidad, carece de funcionalidades específicas para la gestión de memes, como plantillas reutilizables y sistemas de búsqueda avanzada que consideren categorías de memes o contextos humorísticos.
  \item \textbf{\href{https://www.icloud.com/}{iCloud}}, \textbf{\href{https://drive.google.com/drive/my-drive?hl=es-419}{Google Drive}}, \textbf{\href{https://mega.io/es}{Mega}}, \textbf{\href{https://www.microsoft.com/es-es/microsoft-365/onedrive/online-cloud-storage}{OneDrive}} y \textbf{\href{https://www.dropbox.com/}{Dropbox}} son opciones de almacenamiento con funcionalidades completas, aunque ninguna ofrece herramientas especializadas para la gestión y organización de memes. ICloud destaca por su integración en el ecosistema de Apple, permitiendo sincronizar archivos entre dispositivos de la marca. Google Drive y OneDrive, de Google y Microsoft respectivamente, ofrecen un acceso sencillo y compartido a los archivos en la nube. Mega y Dropbox se diferencian al proporcionar una mayor capacidad de almacenamiento gratuito. Sin embargo, en todos los casos, estas plataformas se enfocan en el almacenamiento general de archivos sin opciones específicas para memes.
  \item \textbf{\href{https://nextcloud.com/es/}{Nextcloud}:} Ofrece opciones avanzadas de sincronización y colaboración en la nube, adaptándose parcialmente a la gestión de memes. Sin embargo, su configuración y administración pueden ser complicadas para usuarios sin conocimientos técnicos. Además, carece de herramientas específicas para la edición o categorización de memes, limitando su utilidad como herramienta especializada.
  \item \textbf{\href{https://es.pinterest.com/}{Pinterest}:} Aunque no es una plataforma de almacenamiento en el sentido tradicional, Pinterest permite a los usuarios organizar contenido visual en tableros categorizados, lo cual facilita la agrupación de memes por temas. Sin embargo, no ofrece herramientas para la edición directa de memes, y su estructura no está orientada al almacenamiento profesional o colaborativo de este tipo de contenido.
\end{itemize}

\section{Conclusión de comparativas}

En esta comparación de alternativas con el proyecto, se destacan varios puntos diferenciadores que consolidan esta propuesta como una solución \textbf{innovadora}:

\begin{itemize}
  \item \textbf{Facilidad de acceso y colaboración:} Al ser una aplicación web permite a los usuarios acceder desde cualquier dispositivo y colaborar en la creación y modificación de memes, algo que no ofrecen muchas de las plataformas revisadas.
  \item \textbf{Centralización y organización:} La plataforma incorpora un sistema específico para el almacenamiento y organización de memes en catálogos, incluyendo funcionalidades avanzadas de búsqueda y gestión de plantillas reutilizables. Esto representa una ventaja significativa frente a las soluciones actuales de almacenamiento general.
  \item \textbf{Especialización en memes:} A diferencia de las herramientas de edición de imágenes o plataformas de almacenamiento genéricas, esta aplicación se enfoca principalmente en la creación y gestión de memes. Esto permite una optimización de la experiencia del usuario y la personalización de herramientas específicas para este tipo de contenido, sin limitar la posibilidad de que se suba y organice otro tipo de material visual. Esto asegura que, aunque centrada en los memes, la plataforma sea versátil y pueda adaptarse a otras necesidades de los usuarios.
\end{itemize}